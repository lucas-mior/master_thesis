% https://github.com/SublimeText/LaTeXTools/issues/1439
%!TEX output_directory=.

% You can build this using the command:
% latexmk -pdf -jobname=main \
% -output-directory=. -aux-directory=. \
% -pdflatex="pdflatex -interaction=nonstopmode" -use-make \
% main.tex

% When the bibliography includes a cyclic reference to another bibliography,
% you need to run `pdflatex` 5 times on the following order:
% 1. `pdflatex`,
% 2. `biber`,
% 3. `pdflatex`
% 4. `pdflatex`
% 5. `pdflatex`
% 6. `biber`
% 7. `pdflatex`

% Monograph LaTeX Template for UFSC based on:
% 1. https://github.com/royertiago/tcc
% 2. http://portal.bu.ufsc.br/normalizacao/
% 3. https://github.com/evandrocoan/ufscthesisx
% 4. http://www.latextemplates.com/template/simple-sectioned-essay

% Initially translated from Portuguese with help of
% https://github.com/omegat-org/omegat
% q/computer-assisted-translation-of-latex-document

% Allows you to write your thesis both in English and Portuguese
% q/is-it-possible-to-keep-my-translation-together-with-original-text
\newif\ifenglish\englishfalse{}
\newif\ifadvisor\advisorfalse{}

\pdfcompresslevel=0
\pdfobjcompresslevel=0

% Uncomment `\englishtrue` to set the document default language to English
% \englishtrue
\advisortrue{}

% q/how-to-expand-ifthenelse-so-that-it-can-be-used-in-parshape
\newcommand{\EngAndPor}[2]{%
  \ifenglish#1\else#2\fi%
}
\newcommand{\advisor}[2]{%
  \ifadvisor#1\else#2\fi%
}

% q/how-to-make-passoptionstopackage-add-the-option-as-the-last
% q/changing-the-cleveref-package-language-conjunction-definition
% q/why-isnt-my-biblatex-language-changing-when-passing-the-language-on-my-document
\ifenglish{}
  \PassOptionsToPackage{brazil,main=english}{babel}
\else
  \PassOptionsToPackage{main=brazil,english}{babel}
\fi

% Simple alias for English and Portuguese words
% q/argument-of-bbltempd-has-an-extra
\newcommand{\brazilword}[1]{%
  \protect\foreignlanguage{brazil}{#1}%
}
\newcommand{\englishword}[1]{%
  \protect\foreignlanguage{english}{#1}%
}

% Allow you to write `Evandro's house` in latex as `Evandro\s house` instead of
% `Evandro\textquotesingle{}s house`
% q/space-after-latex-commands
\newcommand{\s}[0]{%
  \textquotesingle{}s\xspace
}
\newcommand{\q}[0]{%
  \textquotesingle{}\xspace
}

% Uncomment the following line if you want to use other biblatex settings
% \PassOptionsToPackage{style=numeric,
%                       repeatfields=true,
%                       backend=biber,
%                       backref=true,
%                       citecounter=true}{biblatex}
\documentclass[%
  % q/changing-the-cleveref-package-language-conjunction-definition
  \EngAndPor{english}{brazilian,brazil},
  12pt,          % Padrão UFSC para versão final
  a4paper,       % Padrão UFSC para versão final
  oneside,       % Impressão nos dois lados da folha
  chapter=TITLE, % Título de capítulos em caixa alta
  section=TITLE, % Título de seções em caixa alta
  % draft,
]{setup/ufscthesisx}

\makenoidxglossaries{}
\UFSCaddSymbol{E_min}{\ensuremath{E_{min}}}{Minimum energy}
\UFSCaddSymbol{alpha}{\ensuremath{\alpha}}{Alpha particle}
\UFSCaddSymbol{F}{\ensuremath{F}}{Force}
\UFSCaddSymbol{rho}{\ensuremath{\rho}}{Density}

\UFSCaddSigla{ABNT}{Brazilian Association of Technical Standards}

% Utilize o arquivo references.bib para incluir sua bibliografia.
% http://tug.ctan.org/tex-archive/macros/latex/contrib/cleveref/cleveref.pdf
\addbibresource{references.bib}

% https://www.overleaf.com/learn/latex/Inserting_Images
\graphicspath{{fig/}}

% FIXME: Preencha com seus dados
\author{\brazilword{Escreva aqui o Nome completo do Autor ou da Autora}}
\title{
  \EngAndPor{%
    Work Title\protect\\Could be break into two lines%
  }{%
    Título do trabalho\protect\\Pode ser Quebrado em duas linhas%
  }%
}

% FIXME: Se houver subtítulo
\UFSCsubtitulo{\EngAndPor{Subtitle}{Subtítulo}}

% FIXME: Siglas para grau de formação
% Dr./Dra., Me./Ma, Bel. Bela. (inglês: PhD., MSc., Bs.)
\ABNTEXorientador[\EngAndPor{Supervisor}{Orientador(a)}]{%
  \brazilword{Nome completo do Orientador(a)}, \EngAndPor{Phd.}{Dr.}%
}

% FIXME: Se houver coorientador, descomente a linha abaixo
% \ABNTEXcoorientador[\EngAndPor{Co-supervisor}{Coorientador(a)}]
% {\brazilword{Nome do coorientador(a)}, \EngAndPor{Phd.}{Dr.}}

% FIXME: Preencher com o nome do Coordenador de TCCs/Teses do seu curso
\UFSCcoordenador[\EngAndPor{Coordinator}{Coordenador(a)}]{%
  \brazilword{Nome do Coordenador(a)}, \EngAndPor{Phd.}{Dr.}%
}

% FIXME: Local da sua defesa
\ABNTEXlocal{%
  \brazilword{Florianópolis, Santa Catarina} -- \EngAndPor{Brazil}{Brasil}%
}

% FIXME: Ano da sua defesa
\UFSCano{2019}
\UFSCbiblioteca{\EngAndPor{University Library}{Biblioteca Universitária}}

% FIXME: Sigla da sua instituição
\UFSCinstituicaoSigla{UFSC}
\ABNTEXinstituicao{\brazilword{Universidade Federal de Santa Catarina}}

% FIXME: Preencha com
% Tese, Dissertação, Trabalho de Conclusão de Curso, Bachelor's Thesis, etc
\ABNTEXtipotrabalho{\EngAndPor{Bachelor\s Thesis}{Trabalho de Conclusão de Curso}}

% FIXME: Se houver Área de Concentração, descomente a linha abaixo
% \UFSCarea{\EngAndPor{Formal Languages}{Linguagens Formais}}

% FIXME: Preencha com Doutor, Bacharel ou Mestrando
\UFSCformacao{%
  \EngAndPor{%
    Bachelor of Science degree in Computer Science%
  }{%
    Bacharel em Ciências da Computação%
  }%
}
\UFSCprograma{%
  \EngAndPor{%
    Undergraduate Program in Computer Science%
  }{%
    Programa de Graduação em Ciências da Computação%
  }%
}

% FIXME: Preencha com Departamento de XXXXXX, Centro de XXXXXX
\UFSCcentro{%
  INE -- %
  \EngAndPor{
    Department of Informatics and Statistics, CTC -- Technological Center%
  }{%
    Departamento de Informática e Estatística, CTC -- Centro Tecnológico%
  }%
}

% FIXME: Preencha com Campus XXXXXX     ou     Centro de XXXXXX
\UFSCcampus{\brazilword{Campus Reitor João David Ferreira Lima}}

% FIXME: Data da sua defesa
\ABNTEXdata{\EngAndPor{30 of march of}{30 de março de} 2019}

% O preambulo deve conter tipo do trabalho, objetivo, nome da instituição e a
% área de concentração.
\ABNTEXpreambulo{%
  \EngAndPor{%
    \abnteximprimirtipotrabalho~submitted to the \imprimirprograma~of
    \abnteximprimirinstituicao~for degree acquirement in \imprimirformacao.%
  }{%
    \abnteximprimirtipotrabalho~submetido ao \imprimirprograma~da
    \abnteximprimirinstituicao~para a obtenção do Grau de \imprimirformacao.%
  }%
}

% Allows you to use ~= instead of `\hyp{}`
% q/how-to-create-an-alternative-to-shortcut-or-hyp
% q/depending-on-babel-language-setting-i-get-biblatex-error-argument-of-language
% q/argument-of-languageactivearg-has-an-extra-i-use-includegraphics-and-r
\useshorthands{~}\defineshorthand{~=}{\hyp{}}

\UFSCpalavrasChave{palavrasChaveIngles}   {Keyword 1}
\UFSCpalavrasChave{palavrasChavePortugues}{Palavra~=Chave 1}

\UFSCpalavrasChave{palavrasChaveIngles}   {Keyword 2}
\UFSCpalavrasChave{palavrasChavePortugues}{Palavra~=Chave 2}

\UFSCpalavrasChave{palavrasChaveIngles}   {Keyword 3}
\UFSCpalavrasChave{palavrasChavePortugues}{Palavra~=Chave 3}

\hypersetup{
  pdfsubject={Thesis' Abstract},
  pdfcreator={LaTeX with abnTeX2 for UFSC},
  pdftitle=\thetitle,
  pdfauthor=\theauthor,
  pdfkeywords={
    \EngAndPor{%
      \palavrasChaveInglessemitem{}
    }{%
      \palavrasChavePortuguessemitem{}
    }
  }
}

% Altere 'settings.tex' para customizar aparência da sua tese
\makeatletter

% Uncomment this if you are debugging pages' badness Underfull & Overflow
% q/geometry-showframe-landscape
% q/what-is-the-difference-between-usepackageshowframe-and-usepackageshowframe
% q/how-to-do-the-memoir-headings-fix-but-not-have-my-text-going-over-the-page-botto
% q/print-page-margins-of-a-document
% \usepackage[showframe,pass]{geometry}

% To use the font Times New Roman, instead of the default LaTeX font
% more up-to-date than '\usepackage{mathptmx}'
% \usepackage{newtxtext}
% \usepackage{newtxmath}

% q/how-to-manually-set-where-a-word-is-split
\hyphenation{Ge-la-im}
\hyphenation{Cis-la-ghi}

% Add missing translations for Portuguese
% q/what-is-the-right-way-to-redefine-macros-defined-by-babel
\@ifpackageloaded{babel}{%
  \@ifpackagewith{babel}{brazil}{%
    \addto\captionsbrazil{%
      \renewcommand{\UFSCtextPreliminaryListName}{%
        Breve Sumário
      }
    }
  }{}
}{}

\@ifundefined{advisor}{\newcommand{\advisor}[2]{#1}}{}

% Selects a sans serif font family
\renewcommand{\sfdefault}{cmss}

% Selects a monospaced (“typewriter”) font family
% \renewcommand{\ttdefault}{cmtt}

% Spacing between lines and paragraphs
% q/ifpackageloaded-question
\@ifclassloaded{memoir}{
  % New custom chapter style VZ14, see other chapters styles in:
  % http://repositorios.cpai.unb.br
  % /ctan/info/latex-samples/MemoirChapStyles/MemoirChapStyles.pdf
  \newcommand{\thickhrulefill}{
    \leavevmode
    \leaders{}
    \hrule height 1ex
    \hfill
    \kern{}
    \z@
  }
  \makechapterstyle{VZ14} {%
    % \thispagestyle{empty}
    \setlength\beforechapskip{50pt}
    \setlength\midchapskip{20pt}
    \setlength\afterchapskip{20pt}
    \renewcommand\chapternamenum{}
    \renewcommand\printchaptername{}
    \renewcommand\chapnamefont{\Huge\scshape}
    \renewcommand\printchapternum{%
      \chapnamefont\null\thickhrulefill\quad
      \@chapapp\space\thechapter\quad\thickhrulefill{}
    }
    \renewcommand\printchapternonum{%
      \par\thickhrulefill\par\vskip\midchapskip{}
      \hrule\vskip\midchapskip{}
    }
    \renewcommand\chaptitlefont{\huge\scshape\centering}
    \renewcommand\afterchapternum{%
      \par\nobreak\vskip\midchapskip\hrule\vskip\midchapskip{}
    }
    \renewcommand\afterchaptertitle{%
      \par\vskip\midchapskip\hrule\nobreak\vskip\afterchapskip{}
    }
  }

  % Apply the style `VZ14` just created
  % \chapterstyle{VZ14}

  % http://mirrors.ibiblio.org/CTAN/macros/latex/contrib/memoir/memman.pdf
  \setlength\beforechapskip{0pt}
  \setlength\midchapskip{15pt}
  \setlength\afterchapskip{15pt}

  % Memoir:
  % Warnings “The material used in the headers is too large” w/ accented titles
  % q/how-to-change-the-page-layout-with-memoir
  \setheadfoot{30.0pt}{\footskip}
  \checkandfixthelayout{}
}{}

% Controlling the spacing between one paragraph and another
% Default value for UFSC 0.0cm
\setlength{\parskip}{\advisor{0.0cm}{0.2cm}}

% Paragraph size is given by
% Default value for UFSC 1.5cm
% \setlength{\parindent}{1.3cm}

% q/how-to-remove-space-before-enumerate
% q/behaviour-of-enumitem-setlist
\advisor{}{
    \setlist*[enumerate]{label=\arabic*,}
    \setlist*[ABNTEXenumerateOptional]{label=\arabic*,}

    % q/space-after-float-with-h
    % q/how-can-i-reduce-padding-after-figure
    \AtBeginEnvironment{figure}{
      % Vertical space above & below [h] floats
      \setlength{\intextsep}{5pt}
      % Vertical space below (above) [t] ([b]) floats
      % \setlength{\textfloatsep}{10pt}
      % \setlength{\abovecaptionskip}{10pt}
      % \setlength{\belowcaptionskip}{5pt}
    }

    % Patch the `abntex2`
    % citacao environment removing the extra space from its top
    % q/topsep-itemsep-partopsep-and-parsep-what-does-each-of-them-mean-and-wha
    \xpatchcmd{\citacao}
    {\list{}}
    {\list{}{\topsep=0pt}}
    {}
    {\FAILEDPATCHINGCITACAO}
}

% Color settings across the document
\@ifpackageloaded{xcolor}{
  % RGB colors in absolute values from 0 to 255 by using `RGB` tag
  \definecolor{darkblue}{RGB}{26,13,178}

  % Colors names definitions
  % as RGB colors in percentage notation by using `rgb` tag
  \definecolor{mygreen}{rgb}{0,0.6,0}
  \definecolor{mygray}{rgb}{0.5,0.5,0.5}
  \definecolor{mymauve}{rgb}{0.58,0,0.82}
  \definecolor{figcolor}{rgb}{1,0.4,0}
  \definecolor{tabcolor}{rgb}{1,0.4,0}
  \definecolor{eqncolor}{rgb}{1,0.4,0}
  \definecolor{linkcolor}{rgb}{1,0.4,0}
  \definecolor{citecolor}{rgb}{1,0.4,0}
  \definecolor{seccolor}{rgb}{0,0,1}
  \definecolor{abscolor}{rgb}{0,0,1}
  \definecolor{titlecolor}{rgb}{0,0,1}
  \definecolor{biocolor}{rgb}{0,0,1}
  \definecolor{blue}{RGB}{41,5,195}

  % PDF Hyperlinks settings
  \@ifpackageloaded{hyperref}{
    \hypersetup{
      colorlinks=true,  % false: boxed links; true: colored links
      linkcolor=black,  % color of internal links
      citecolor=black,  % color of links to bibliography
      filecolor=black,  % color of file links
      urlcolor=black,
      bookmarksdepth=4,
      pdfencoding=auto,%
      psdextra,
    }
  }
}{}

% Filtering and Mapping Bibliographies
% \DeclareFieldFormat{url}{Disponível~em:\addspace\url{#1}}

% q/how-to-make-biblatex-url-links-generated-with-brackets-around-it-url-correctly
\DeclareFieldFormat{url}{%
  \bibstring{urlfrom}%
  \addcolon%
  \space%
  \textless%
  \url{#1}%
  \textgreater{}
}
\DefineBibliographyStrings{brazil}{urlfrom = {Disponível em}}
\DefineBibliographyStrings{english}{urlfrom = {Available from}}

% q/is-possible-to-remove-the-link-color-of-the-comma-on-the-citation-link
% \DeclareFieldFormat{citehyperref}{#1}

% % q/reduce-font-size-of-bibliography-overfull-bibliography
% \newcommand{\bibliographyfontsize}{\fontsize{10.0pt}{10.5pt}\selectfont}
% \renewcommand*{\bibfont}{\bibliographyfontsize}

% Uncomment this to insert the abstract
% into your bibliography entries when the abstract is available
% q/how-to-correctly-insert-and-justify-abstract
\ifadvisor\else
  \DeclareFieldFormat{abstract}{%
    \par\justifying{}
    \begin{adjustwidth}{1cm}{}
      \textbf{\bibsentence\bibstring{abstract}:} #1
    \end{adjustwidth}
  }
  \renewbibmacro*{finentry}{%
    \iffieldundef{abstract}
    {\finentry}
    {%
      \finentrypunct{}
      \printfield{abstract}%
      \renewcommand*{\finentrypunct}{}%
      \finentry{}
    }
  }

  % Backref package settings, pages with citations in bibliography
  \newcommand{\biblatexcitedntimes}{%
    \autocap{c}ited \arabic{citecounter} times
  }
  \newcommand{\biblatexcitedonetime}{%
    \autocap{c}ited one time
  }
  \newcommand{\biblatexcitednotimes}{%
    \autocap{n}o citation in the text
  }

  \@ifpackageloaded{babel}{%
    \@ifpackagewith{babel}{brazil}{%
      \addto\captionsbrazil{%
        \renewcommand{\biblatexcitedntimes}{%
          \autocap{c}itado \arabic{citecounter} vezes
        }
        \renewcommand{\biblatexcitedonetime}{%
          \autocap{c}itado uma vez
        }
        \renewcommand{\biblatexcitednotimes}{%
          \autocap{n}enhuma citação no texto
        }
      }
    }{}
  }{}

  \@ifpackageloaded{biblatex}{%
    % q/how-to-detect-whether-the-option-citecounter-was-enabled-on-biblatex
    \ifx\blx@citecounter\relax
      \message{Is citecounter defined? NO!^^J}
    \else
      \message{Is citecounter defined? YES!^^J}
      \ifbacktracker{}
        \message{Is backtracker defined? YES!^^J}
        \renewbibmacro*{pageref}{%
          % q/how-to-use-a-dot-to-separate-my-new-bibliography-entry
          \renewcommand*{\bibpagerefpunct}{\addperiod\space}%
          \iflistundef{pageref}{%
            \printtext{\biblatexcitednotimes}
          }{%
            \printtext{%
              \ifnumgreater{\value{citecounter}}{1}{%
                \biblatexcitedntimes{}
              }{%
                \biblatexcitedonetime{}
              }%
            }%
            \setunit{\addspace}%
            \ifnumgreater{\value{pageref}}{1}{
              \bibstring{backrefpages}\ppspace{}
            }{%
              \bibstring{backrefpage}\ppspace{}
            }%
            \printlist[pageref][-\value{listtotal}]{pageref}%
          }%
        }

        \DefineBibliographyStrings{brazil}{
          backrefpage  = {na página},
          backrefpages = {nas páginas},
        }

        \DefineBibliographyStrings{english}{
          backrefpage  = {on page},
          backrefpages = {on pages},
        }
      \else
        \message{Is backtracker defined? NO!^^J}
      \fi
    \fi
  }{}
\fi

% q/why-an-empty-or-not-biblatex-declaresourcemap-is-removing-my-bibliography-acces
% https://github.com/abntex/biblatex-abnt/pull/56/files
\DeclareStyleSourcemap{% >>>2
  % This maps some fields used in abntex2cite to biblatex fields.
  \maps[datatype=bibtex]{%
    \map{%
      \step[fieldsource=conference-number,fieldtarget=number]%
      \step[fieldsource=conference-year,fieldtarget=eventdate]%
      \step[fieldsource=conference-location,fieldtarget=venue]%
      \step[fieldsource=conference-number,fieldtarget=number]%
      \step[fieldsource=org-short,fieldtarget=shortauthor]%
      \step[fieldsource=urlaccessdate,fieldtarget=urldate]%
      \step[fieldsource=year-presented,fieldtarget=eventyear]%
      \step[fieldsource=furtherresp,fieldtarget=titleaddon]%
      \step[typesource=journalpart,typetarget=supperiodical]%
    }%
    \map[overwrite=false]{%
      \step[fieldsource=reprinted-from, final]%
      \step[fieldset=related, origfieldval]%
    }%
    \map[overwrite=false]{%
      \step[fieldsource=reprinted-text, final]%
      \step[fieldset=relatedtype, fieldvalue={reprintfrom}]%
    }%
    \map{%
      \pertype{patent}% Use the organization as sourcekey for patents
      \step[fieldsource=organization, final]%
      \step[fieldset=sortkey, origfieldval]%
    }%
    \map[overwrite=false]{%
      \pertype{thesis}%
      \pertype{phdthesis}%
      \pertype{mastersthesis}%
      \pertype{monography}%
      \step[fieldset=bookpagination, fieldvalue={sheet}]%
    }%
    % remove fields that are always useless
    \map{
      % \step[fieldset=abstract, null]
      \step[fieldset=pagetotal, null]
    }
    % % remove URLs for types that are primarily printed
    % \map{
    %   \pernottype{software}
    %   \pernottype{online}
    %   \pernottype{report}
    %   \pernottype{techreport}
    %   \pernottype{standard}
    %   \pernottype{manual}
    %   \pernottype{misc}
    %   \step[fieldset=url, null]
    %   \step[fieldset=urldate, null]
    % }
    \map{
      \pertype{inproceedings}
      % remove mostly redundant conference information
      \step[fieldset=venue, null]
      \step[fieldset=eventdate, null]
      \step[fieldset=eventtitle, null]
      % do not show ISBN for proceedings
      \step[fieldset=isbn, null]
      % Citavi bug
      \step[fieldset=volume, null]
    }
  }%
}% <<<2

% q/changing-the-font-of-the-numbers-in-the-toc-in-the-memoir-class
\renewcommand{\cftpartfont}{%
  \ABNTEXpartfont\color{black}%
}
\renewcommand{\cftpartpagefont}{%
  \ABNTEXpartfont\color{black}%
}
\renewcommand{\cftchapterfont}{%
  \ABNTEXchapterfont\color{black}%
}
\renewcommand{\cftchapterpagefont}{%
  \ABNTEXchapterfont\color{black}%
}
\renewcommand{\cftsectionfont}{%
  \ABNTEXsectionfont\color{black}%
}
\renewcommand{\cftsectionpagefont}{%
  \ABNTEXsectionfont\color{black}%
}
\renewcommand{\cftsubsectionfont}{%
  \ABNTEXsubsectionfont\color{black}%
}
\renewcommand{\cftsubsectionpagefont}{%
  \ABNTEXsubsectionfont\color{black}%
}
\renewcommand{\cftsubsubsectionfont}{%
  \ABNTEXsubsubsectionfont\color{black}%
}
\renewcommand{\cftsubsubsectionpagefont}{%
  \ABNTEXsubsubsectionfont\color{black}%
}
\renewcommand{\cftparagraphfont}{%
  \ABNTEXsubsubsubsectionfont\color{black}%
}
\renewcommand{\cftparagraphpagefont}{%
  \ABNTEXsubsubsubsectionfont\color{black}%
}

% Memoir has another mechanism for the job:
% \cftsetindents{‹kind›}{indent}{numwidth}. Here kind is chapter, section, or
% whatever; the indent specifies the ‘margin’ before the entry starts; and the
% width is of the box into which the number is typeset (so needs to be wide
% enough for the largest number, with the necessary spacing to separate it from
% what comes after it in the line.
% http://www.tex.ac.uk/FAQ-tocloftwrong.html
% q/memoir-indentation-of-unnumbered-sections-in-table-of-contents
% q/memoir-toc-indent-the-second-line-by-numberspace
%
% `\ABNTEXcftLastNumWidth` and these `\cftsetindents` are defined by the abntex2
% class, obeying the `ABNTEXsumario-abnt-6027-2012`.
% \newlength{\ABNTEXcftLastNumWidth}
% \setlength{\ABNTEXcftLastNumWidth}{\cftsubsubsectionnumwidth}
% \addtolength{\ABNTEXcftLastNumWidth}{-1em}

% http://www.tex.ac.uk/FAQ-tocloftwrong.html
% Use \setlength\cftsectionnumwidth{4em} to override all these values at once
\ifadvisor\else
  \makechapterstyle{fixedabntex2indentation}{%
    \renewcommand{\chapterheadstart}{}
    \setlength{\beforechapskip}{20pt}
    \setlength{\midchapskip}{20pt}
    \setlength{\afterchapskip}{15pt}

    \ifx\ABNTEXchapterNameNumLength\undefined{}
      \newlength{\ABNTEXchapterNameNumLength}
    \fi

    % tamanhos de fontes de chapter e part
    \ifthenelse{\equal{\ABNTEXisarticle}{true}}{%
      \setlength\beforechapskip{\baselineskip}%
      \renewcommand{\chaptitlefont}{\ABNTEXsectionfont\ABNTEXsectionfontsize}%
    }{
      \setlength{\beforechapskip}{0pt}%
      \renewcommand{\chaptitlefont}{\ABNTEXchapterfont\ABNTEXchapterfontsize}%
    }

    \renewcommand{\chapnumfont}{\chaptitlefont}
    \renewcommand{\parttitlefont}{\ABNTEXpartfont\ABNTEXpartfontsize}
    \renewcommand{\partnumfont}{\ABNTEXpartfont\ABNTEXpartfontsize}
    \renewcommand{\partnamefont}{\ABNTEXpartfont\ABNTEXpartfontsize}

    % tamanhos de fontes de
    % section, subsection, subsubsection e subsubsubsection
    \setsecheadstyle{
      \ABNTEXsectionfont%
      \ABNTEXsectionfontsize%
      \ABNTEXsectionupperifneeded{}
    }
    \setsubsecheadstyle{
      \ABNTEXsubsectionfont%
      \ABNTEXsubsectionfontsize%
      \ABNTEXsubsectionupperifneeded{}
    }
    \setsubsubsecheadstyle{
      \ABNTEXsubsubsectionfont%
      \ABNTEXsubsubsectionfontsize%
      \ABNTEXsubsubsectionupperifneeded{}
    }
    \ABNTEXsetsubsubsubsecheadstyle{
      \ABNTEXsubsubsubsectionfont%
      \ABNTEXsubsubsubsectionfontsize%
      \ABNTEXsubsubsubsectionupperifneeded{}
    }

    % Impressão do número do capítulo
    \renewcommand{\chapternamenum}{}

    % Impressão do nome do capítulo
    \renewcommand{\printchaptername}{%
      \chaptitlefont%
      \ifthenelse{\boolean{abntex@apendiceousecao}}{%
        \appendixname%
      }{}%
    }

    % Impressão do título do capítulo
    \def\printchaptertitle##1{%
      \chaptitlefont%
      \ifthenelse{\boolean{abntex@innonumchapter}}{%
        \centering\ABNTEXchapterupperifneeded{##1}
      }{%
        \ifthenelse{\boolean{abntex@apendiceousecao}}{%
          \centering%
          \settowidth{\ABNTEXchapterNameNumLength}{%
            \printchaptername%
            \printchapternum%
            \afterchapternum%
          }%
          \ABNTEXchapterupperifneeded{##1}%
        }{%
          \settowidth{\ABNTEXchapterNameNumLength}{%
            \printchaptername%
            \printchapternum%
            \afterchapternum%
          }%
          \parbox[t]{\columnwidth-\ABNTEXchapterNameNumLength}{%
            \ABNTEXchapterupperifneeded{##1}}%
          }%
      }%
    }

    % q/memoir-indentation-of-unnumbered-sections-in-table-of-contents
    \renewcommand{\ABNTEXtocinnonumchapter}{%
      \addtocontents{toc}{\cftsetindents{chapter}{2.5em}{2em}}%
      \cftinserthook{toc}{A}}

    % Impressão do número do capítulo (no capítulo e não toc)
    \renewcommand{\printchapternum}{%
      \setboolean{abntex@innonumchapter}{false}%
      \chapnumfont%
      ~\thechapter~%chktex 39
      \ifthenelse{\boolean{abntex@apendiceousecao}}{%
        \ABNTEXtocinnonumchapter%
        ~\ABNTEXcaptiondelim~%chktex 39
      }{}%
    }

    \renewcommand{\ABNTEXcaptiondelim}{~\textendash~}
    \renewcommand{\afterchapternum}{}

    % Impressão do capítulo não numerado
    \renewcommand\printchapternonum{%
      \setboolean{abntex@innonumchapter}{true}%
    }
  }
  \chapterstyle{fixedabntex2indentation}

  \cftsetindents{part}          {0em} {3em}
  \cftsetindents{chapter}       {0em} {3em}
  \cftsetindents{section}       {0em} {4.3em}
  \cftsetindents{subsection}    {0em} {5.2em}
  \cftsetindents{subsubsection} {0em} {5.1em}
  \cftsetindents{paragraph}     {0em} {6.0em}
  \cftsetindents{subparagraph}  {0em} {7.0em}
\fi

\makeatother


% When writing a large document, it is sometimes useful to work on selected
% sections of the document to speed up compilation time:
% https://en.wikibooks.org/wiki/TeX/includeonly
\newif\ifforcedinclude\forcedincludefalse{}

% \addtoincludeonly{abntexEnvAgradecimentos}
% \addtoincludeonly{abntexEnvEpigrafe}
% \addtoincludeonly{abntexEnvFichacatalografica}
% \addtoincludeonly{abntexEnvFolhadeaprovacao}
% \addtoincludeonly{resumos}
% \addtoincludeonly{abntexEnvSiglas}
% \addtoincludeonly{abntexEnvSimbolos}

% Part 1
% \addtoincludeonly{introduction}
% \addtoincludeonly{motivation}
% \addtoincludeonly{beautifiers}

% Part 2
% \addtoincludeonly{conclusion}

% Control whether the full document will be generated
% Note: It will also generate severals errors like the following, can be ignored
%       Latexmk: Missing input file: 'test.aux'
%
% You can make latex stop generate these errors, if you generate a full version
% of the document, before uncommenting these lines.
%
% Uncomment these lines to only partially generate the document
% \ifnum\StrLen{\includeonlyfilelist}>0\relax
%   \includeonly{\includeonlyfilelist}
% \fi
% \forcedincludetrue

% q/xrightarrow-text
\makeatletter
\newcommand{\xRightarrow}[2][]{%
  \ext@arrow0359\Rightarrowfill@{#1}{#2}%
}
\newcommand{\xLeftarrow}[2][]{%
  \ext@arrow0359\Leftarrowfill@{#1}{#2}%
}
\makeatother

% q/footnote-runs-onto-second-page
\interfootnotelinepenalty=10000

% Disable the empty pages automatically put by memoir class,
% except the ones by \cleardoublepage
\ifforcedinclude{}
  \openany{}
\fi

% q/overfull-hbox-in-biblatex
% q/why-my-document-is-not-hyphenation-on-words-starting-with-upper-case-letter-i
\emergencystretch=5em

% q/how-can-i-reduce-padding-after-figure
% q/how-to-keep-my-default-floating-environment-spacing-before-them-while-reducing
% \xpretocmd{\figure}{\setlength{\belowcaptionskip}{-10pt}}{}{}

\renewcommand\familydefault{\sfdefault}
\renewcommand{\sfdefault}{cmss}

% Extract only the basename (no directories, no extension)
\usepackage{xstring}
\usepackage{subcaption}

% #1 = filename (with path)
% #2 = caption
% #3 = scale (e.g., 0.35)
\newcommand{\image}[3]{%
  % Remove path
  \StrBehind{#1}{/}[\tmpA]%
  \IfEq{\tmpA}{}{%
    \def\tmpA{#1}%
  }{}%
  % Remove extension
  \StrBefore{\tmpA}{.}[\imglabel]%
  %
  \begin{figure}[!htb]
    \centering
    \includegraphics[scale=#3]{#1}
    \caption{#2}%
    \label{fig:\imglabel}
  \end{figure}
}

% #1 = arquivo 1
% #2 = legenda 1
% #3 = largura relativa 1
% #4 = arquivo 2
% #5 = legenda 2
% #6 = largura relativa 2
% #7 = legenda geral
\newcommand{\duasimages}[7]{%
  \StrBehind{#1}{/}[\tmpA]%
  \IfEq{\tmpA}{}{%
    \def\tmpA{#1}%
  }{}%
  \StrBefore{\tmpA}{.}[\imglabelA]%

  \StrBehind{#4}{/}[\tmpB]%
  \IfEq{\tmpB}{}{%
    \def\tmpB{#4}%
  }{}%
  \StrBefore{\tmpB}{.}[\imglabelB]%

  \edef\mainlabel{fig:\imglabelA-\imglabelB}%

  \begin{figure}[htb]
    \centering

    \begin{subfigure}[t]{0.48\textwidth}
      \centering
      \includegraphics[width=#3\linewidth]{#1}
      \caption{#2}%
      \label{fig:\imglabelA}
    \end{subfigure}
    \hfill
    \begin{subfigure}[t]{0.48\textwidth}
      \centering
      \includegraphics[width=#6\linewidth]{#4}
      \caption{#5}%
      \label{fig:\imglabelB}
    \end{subfigure}

    \caption{#7}%
    \label{\mainlabel}
  \end{figure}
}

\begin{document}
  % FIXME: Comment this after finishing the thesis,
  %        so you can start fixing the \flushbottom vs \raggedbottom
  % q/flushbottom-vs-raggedbottom
  \raggedbottom{}

  % q/double-space-between-sentences
  \frenchspacing

  % Uncomment this to put a ←← | ← (Go To Top/Go Back) on each section header
  \ifadvisor\else
    \addGoToSummary{}
  \fi

  % ELEMENTOS PRÉ-TEXTUAIS
  \ifforcedinclude\else
    % Fix the \textpreliminarycontents not showing up when @twoside is off
    \newif\ifufscThesisXisMemoirTwoSidesEnabled{}

    % q/how-do-i-check-if-a-document-is-oneside-or-twoside
    \ifthenelse{\boolean{@twoside}}{%
      \ufscThesisXisMemoirTwoSidesEnabledtrue%
    }{%
      \ufscThesisXisMemoirTwoSidesEnabledfalse%
    }%
    \setboolean{@twoside}{true}

    % pretextual settings
    % q/how-to-fix-destination-with-the-same-identifier-namepage-a-has-been-already
    % q/pdftex-warning-has-been-referenced-but-does-not-exist-replaced-by-a-fixed-one
    \hypersetup{pageanchor=false}
    \ABNTEXPRIVATEbookmarkthis{Capa}
    \UFSCaddToTextPreliminaryContent{Capa}
    \ABNTEXpretextual{}

    % \includepdf{fig/FrenteCapaUFSC.pdf}
    % q/blank-page-after-titlingpage
    \ifadvisor\else
      \AtBeginShipoutNext{\AtBeginShipoutNext{\AtBeginShipoutDiscard}}
    \fi
    \ABNTEXimprimircapa{}

    % q/how-to-fix-destination-with-the-same-identifier-namepage-a-has-been-already
    % q/pdftex-warning-has-been-referenced-but-does-not-exist-replaced-by-a-fixed-one
    \hypersetup{pageanchor=true}

    % Custom list throw LaTeX Error:
    % Command \mycustomfiction already defined?
    % q/custom-list-throw-latex-error-command-mycustomfiction-already-defined
    \ifadvisor\else
      % Manually add the `\textpreliminarycontents` to the Table of
      % Contents here to keep the hyper link pointing to the beginning of
      % the page, instead of the beginning of `\textpreliminarycontents`
      % q/when-do-i-need-to-invoke-phantomsection
      \phantomsection\addcontentsline{toc}{chapter}
        {\UFSCtextPreliminaryListName}

      % q/list-of-figures-and-tables-when-there-are-no-figures-or-tables
      \whenlistisnotempty{\UFSCtextPreliminaryListName}{%
        \begin{KeepFromToc}
          \textpreliminarycontents{}
        \end{KeepFromToc}
      }

      \clearpage
    \fi

    % Fix the \textpreliminarycontents not showing up when @twoside is off
    \ifufscThesisXisMemoirTwoSidesEnabled{}
      \setboolean{@twoside}{true}
    \else
      \setboolean{@twoside}{false}
    \fi

    % Folha de rosto (o * indica que haverá a ficha bibliográfica)
    % q/table-of-contents-incorrect-page-numbering
    \UFSCaddToTextPreliminaryContent{\ABNTEXfolhaderostoname}
    \ABNTEXimprimirFolhaDeRosto*{}

    % Inserir a ficha bibliografica
    %
    % Isto é um exemplo de Ficha Catalográfica,
    % ou ``Dados internacionais de % catalogação-na-publicação''.
    % Você pode utilizar este modelo como referência.  Porém, provavelmente
    % a biblioteca da sua universidade lhe fornecerá um PDF com a ficha
    % catalográfica definitiva após a defesa do trabalho. Quando estiver com
    % o documento, salve-o como PDF no diretório do seu projeto e substitua
    % todo o conteúdo de implementação deste arquivo pelo comando abaixo:
    \ABNTEXPRIVATEbookmarkthis{Ficha Catalográfica}
    \UFSCaddToTextPreliminaryContent{Ficha Catalográfica}

    \EngAndPor{
  Legal Notes:
  
  There is no warranty for any part of the documented software. The authors have
  taken care in the preparation of this thesis, but make no expressed or implied
  warranty of any kind and assume no responsibility for errors or omissions. No
  liability is assumed for incidental or consequential damages in connection
  with or arising out of the use of the information or programs contained here.
}{
  Notas legais:
  
  Não há garantia para qualquer parte do software documentado. Os autores
  tomaram cuidado na preparação desta tese, mas não fazem nenhuma garantia
  expressa ou implícita de qualquer tipo e não assumem qualquer responsabilidade
  por erros ou omissões. Não se assume qualquer responsabilidade por danos
  incidentais ou consequentes em conexão ou decorrentes do uso das informações
  ou programas aqui contidos.
}

% http://portalbu.ufsc.br/ficha
% http://portal.bu.ufsc.br/servicos/ficha-de-identificacao-da-obra/
\begin{abntexEnvFichacatalografica}
  \vspace*{\fill}

  \begin{center}

    \EngAndPor{%
      Cataloging at source by the
      University Library of the Federal University of Santa Catarina.
    }{%
      Catalogação na fonte pela Biblioteca Universitária da Universidade
      Federal de Santa Catarina.
    }

    \EngAndPor{%
      File compiled at \currenttime{} h of the day \today.
    }{%
      Arquivo compilado às \currenttime{} h do dia \today.
    }

    \framebox[\textwidth]{
      % q/use-the-value-of-title-with-removed-linebreak
      \begin{minipage}{0.98\textwidth}
      \begingroup \let\\=\space

        \ttfamily
        \theauthor{}

        \hspace{0.5cm} \thetitle%
        \UFSCifNotEmpty{\imprimirsubtitulo}{%
          ~:~\imprimirsubtitulo% chktex 39
        }%
        ~/~\theauthor% chktex 39
        {;}~\abnteximprimirorientadorRotulo,~\abnteximprimirorientador%
        \UFSCifNotEmpty{%
          \abnteximprimircoorientador%
        }{%
          {;}~\abnteximprimircoorientadorRotulo,~\abnteximprimircoorientador%
        }%
        ~--~\abnteximprimirlocal,~\ABNTEXimprimirdata.%chktex 39

        % Prints how much pages there are on the document
        % and links to the last page
        \hspace{0.5cm}~\pageref{LastPage} p.
        \bigskip

        \hspace{0.5cm}
        \abnteximprimirtipotrabalho~--~\abnteximprimirinstituicao,
        \imprimircentro,~\imprimirprograma.
        \bigskip

        \hspace{0.5cm} \EngAndPor{Includes references}{Inclui referências}
        \bigskip

        % q/using-lower-case-roman-numerals-in-enumerate-lists
        % q/how-to-define-inparaenum-in-the-preamble
        \hspace{0.5cm}
        \begin{inparaenum}
          \EngAndPor{%
            \palavrasChaveInglescomvirgula%
          }{%
            \palavrasChavePortuguescomvirgula%
          }
        \end{inparaenum}%
        \begin{inparaenum}[I.]
          \item \abnteximprimirorientador{}
          \UFSCifNotEmpty{%
            \abnteximprimircoorientador%
          }{%
            \item \abnteximprimircoorientador~%
          }
          \item \imprimirprograma{}
          \item \thetitle{}
        \end{inparaenum}%
        \bigskip

        \hspace{7.75cm} CDU 02:141:005.7

      \endgroup
      \end{minipage}
    }

  \end{center}

\end{abntexEnvFichacatalografica}

    % q/how-to-include-multiple-pages-in-latex
    % \includepdf{fig/Ficha_Catalografica.pdf}
    \ifforcedinclude\else
      \cleardoublepage{}
    \fi
  \fi

  % Inserir errata

  % Inserir folha de aprovação. Isto é um exemplo de Folha de aprovação,
  % elemento obrigatório da NBR 14724/2011 (seção 4.2.1.3). Você pode utilizar
  % este modelo até a aprovação do trabalho.  Após isso, substitua todo o
  % conteúdo deste arquivo por uma imagem da página assinada pela banca com o
  % comando abaixo:
  \ifforcedinclude\else
    \cleardoublepage{}
  \fi
  \UFSCaddToTextPreliminaryContent{\EngAndPor{Approval Sheet}{Folha de Aprovação}}

\begin{abntexEnvFolhadeaprovacao}

  \begin{center}
    \theauthor{}

    \begin{center}
      {%
        {\ABNTEXchapterfont\MakeUppercase{\thetitle}}%
        \UFSCifNotEmpty{\imprimirsubtitulo}{%
          {\ABNTEXchapterfont:} \imprimirsubtitulo%
        }
      }

    \end{center}

    \begin{minipage}{\textwidth}
      \EngAndPor{
        This \abnteximprimirtipotrabalho~was
        considered appropriate to get the \imprimirformacao,
        \UFSCifNotEmpty{\imprimirarea}{%
          in the area of \imprimirarea,%
        }
        and it was approved by the
        \imprimirprograma~of \imprimircentro~of \abnteximprimirinstituicao.
      }{
        Este \abnteximprimirtipotrabalho~foi
        julgado adequado para obtenção do Título de \imprimirformacao{},
        \UFSCifNotEmpty{\imprimirarea}{%
          na área de concentração de \imprimirarea,%
        } e foi aprovado em sua forma final pelo
        \imprimirprograma~do \imprimircentro~da \abnteximprimirinstituicao.
      }
    \end{minipage}%
  \end{center}

  \begin{center}
    \abnteximprimirlocal, \ABNTEXimprimirdata.
  \end{center}

  \ABNTEXassinatura{%
    \textbf{\imprimircoordenador} \\
    \imprimircoordenadorRotulo~\EngAndPor{of}{do} \imprimirprograma{}
  }

  % \newpage
  \begin{flushleft}
    \textbf{\EngAndPor{Examination Board}{Banca Examinadora}:}
  \end{flushleft}

  \ABNTEXassinatura{%
    \textbf{\abnteximprimirorientador} \\ \abnteximprimirorientadorRotulo\\
    \abnteximprimirinstituicao~--~\abnteximprimirinstituicaosigla{}
  }

  \UFSCifNotEmpty{\abnteximprimircoorientador}{%
    \ABNTEXassinatura{%
      \textbf{\abnteximprimircoorientador}
      \abnteximprimircoorientadorRotulo{} \\
      \abnteximprimirinstituicao~--~\abnteximprimirinstituicaosigla{}
    }
  }

  \ABNTEXassinatura{%
    \textbf{Prof.\ Convidado 1, \EngAndPor{PhD.}{Dr.}} \\
    Instituição 1 -- Sigla 1
  }

  \ABNTEXassinatura{%
    \textbf{Prof.\ Convidado 2, \EngAndPor{PhD.}{Dr.}} \\
    Instituição 2 -- Sigla 2
  }

  \ABNTEXassinatura{%
    \textbf{Prof.\ Convidado 3, \EngAndPor{PhD.}{Dr.}} \\
    Instituição 3 -- Sigla 3
  }

  \ABNTEXassinatura{%
    \textbf{Prof.\ Convidado 4, \EngAndPor{PhD.}{Dr.}} \\
    Instituição 4 -- Sigla 4
  }

\end{abntexEnvFolhadeaprovacao}

  % \includepdf{fig/folhadeaprovacao_final.pdf}

  \ifforcedinclude\else
    \cleardoublepage{}
  \fi
  \ifforcedinclude\else
    \UFSCaddToTextPreliminaryContent{\EngAndPor{Dedicatory}{Dedicatória}}

\begin{abntexEnvDedicatoria}

  \vspace*{\fill}
  \centering
  \noindent
  \textit{\EngAndPor{
    This work is dedicated to adult children who, \\
    When small, dreamed of becoming scientists.
  }{
    Este trabalho é dedicado às crianças adultas que,\\
    quando pequenas, sonharam em se tornar cientistas.
  }}
  \vspace*{\fill}

\end{abntexEnvDedicatoria}

  \fi

  \ifforcedinclude\else
    \cleardoublepage{}
  \fi
  \include{06_agradecimentos.tex}

  \ifforcedinclude\else
    \cleardoublepage{}
  \fi
  \addtotextpreliminarycontent{\lang{Epigraph}{Epigrafe}}

\begin{abntexEnvEpigrafe}

\vspace*{\fill}\lang
{
    \begin{flushright}
        \textit{``Learn from yesterday, live for today, hope for tomorrow. The
                important thing is not to stop questioning.''} \\ Albert Einstein
    \end{flushright}
    \begin{flushright}
        \textit{``The true sign of intelligence is not knowledge but
                imagination.''} \\  Albert Einstein
    \end{flushright}
    \begin{flushright}
        \textit{``Peace cannot be kept by force; it can only be achieved by
                understanding.''} \\ Albert Einstein
    \end{flushright}
    \begin{flushright}
        \textit{``Whoever is careless with the truth in small matters cannot be
                trusted with important matters.''} \\ Albert Einstein
    \end{flushright}
    \begin{flushright}
        \textit{``Extraordinary claims require extraordinary evidence''} \\ Carl Sagan
    \end{flushright}
    \begin{flushright}
        \textit{``Catholic, which I was until I reached the age of reason.''} \\ George Carlin
    \end{flushright}
    \begin{flushright}
        \textit{``We made too many wrong mistakes.''} \\ Yogi Berra
    \end{flushright}
}
{
    \begin{flushright}
        \textit{``Aprenda com o ontem, viva para o hoje, tenha esperança para o amanhã. 
        O importante é não parar de questionar.''} \\ Albert Einstein
    \end{flushright}
    \begin{flushright}
        \textit{``O verdadeiro sinal de inteligência não é o conhecimento, mas a imaginação.''} \\ Albert Einstein
    \end{flushright}
    \begin{flushright}
        \textit{``A paz não pode ser mantida pela força; ela só pode ser alcançada pelo entendimento.''} \\ Albert Einstein
    \end{flushright}
    \begin{flushright}
        \textit{``Quem é descuidado com a verdade em pequenas coisas não pode ser confiável em assuntos importantes.''} \\ Albert Einstein
    \end{flushright}
    \begin{flushright}
        \textit{``Afirmações extraordinárias exigem evidências extraordinárias.''} \\ Carl Sagan
    \end{flushright}
    \begin{flushright}
        \textit{``Católico, o que eu fui até chegar à idade da razão.''} \\ George Carlin
    \end{flushright}
    \begin{flushright}
        \textit{``Cometemos muitos erros errados.''} \\ Yogi Berra
    \end{flushright}
}

\end{abntexEnvEpigrafe}


  % Ajusta o espaçamento dos parágrafos do resumo
  \setlength{\absparsep}{18pt}

  % RESUMOS
  \ifforcedinclude\else
    \cleardoublepage{}
  \fi
  \newcommand{\imprimirbrazilabstract}{%
  \cleardoublepage{}
  \phantomsection{}
  \UFSCaddToTextPreliminaryContent{Resumo em Português}
  \begin{otherlanguage*}{brazil}
    \begin{abntexEnvResumo}[Resumo]

      Segundo a \textcite[3.1-3.2]{NBR6028:2003}, o resumo deve ressaltar o
      objetivo, o método, os resultados e as conclusões do documento. A ordem
      e a extensão destes itens dependem do tipo de resumo (informativo ou
      indicativo) e do tratamento que cada item recebe no documento original.
      O resumo deve ser precedido da referência do documento, com exceção do
      resumo inserido no próprio documento. (\ldots) As palavras-chave devem
      figurar logo abaixo do resumo, antecedidas da expressão Palavras-chave:,
      separadas entre si por ponto e finalizadas também por ponto.

      Além disso, na UFSC o texto do resumo deve ser digitado, em um único
      bloco, sem espaço de parágrafo. O resumo deve ser significativo,
      composto de uma sequência de frases concisas, afirmativas e não de uma
      enumeração de tópicos. Não deve conter citações. Deve usar o verbo na
      voz passiva. Abaixo do resumo, deve-se informar as palavras-chave
      (palavras ou expressões significativas retiradas do texto) ou, termos
      retirados de thesaurus da área.

      \UFSCimprimirPalavrasChave{Palavras-chaves}
      {%
        \begin{inparaitem}[]%
          \palavrasChavePortugues%
        \end{inparaitem}%
      }

    \end{abntexEnvResumo}
  \end{otherlanguage*}
}

\newcommand{\imprimirenglishabstract}{%
  % q/changing-babel-package-inside-a-single-chapter
  % q/multiple-language-document-babel-selectlanguage-vs-begin-endotherlanguage
  \cleardoublepage{}
  \phantomsection{}
  \UFSCaddToTextPreliminaryContent{English's Abstract}
  \begin{otherlanguage*}{english}
  \begin{abntexEnvResumo}[Abstract]

      This is the English abstract.

      \UFSCimprimirPalavrasChave{Keywords}
      {\begin{inparaitem}[]\palavrasChaveIngles\end{inparaitem}}

  \end{abntexEnvResumo}
  \end{otherlanguage*}
}

\makeatletter
\ifenglish{}
  \@ifundefined{imprimirbrazilabstract}{}{\imprimirbrazilabstract}

  % q/times-new-roman-in-latex-just-some-text
  % q/how-to-force-output-to-a-left-or-right-page
  % q/do-not-display-chapter-title-in-memoir-class
  \cleardoublepage{}
  \phantomsection{}
  \ABNTEXpretextualchapter{Resumo Expandido}
  \UFSCaddToTextPreliminaryContent{Resumo Expandido}

  \begin{otherlanguage*}{brazil}
    \setlength{\parskip}{0.2cm}
    \setlength{\parindent}{0.0cm}
    \fontfamily{ptm}\selectfont

    \section*{Introdução}
    O resumo expandido é previsto na Resolução Normativa nº 95/CUn/2017,
    Art. 55, § 2, de 4 de abril de 2017, e exigido para teses e dissertações
    escritas em idiomas estrangeiros (com exceção dos cursos pertinentes ao
    estudo de idiomas estrangeiros – Programa de Pós-Graduação em Estudos da
    Tradução e Programa de Pós-Graduação em Inglês: Estudos Linguísticos e
    Literários).

    O resumo expandido é considerado um elemento pré-textual e deverá ser
    incluído no trabalho após o resumo e antes do abstract. Deverá iniciar
    em página impar (no anverso de uma folha) continuando no verso da folha.
    O texto deverá seguir o formato A5, com margens espelhadas: superior 2,0
    cm, inferior 1,5 cm, interna 2,5 cm e externa 1,5. Deve ser empregada a
    fonte Time New Roman.  Todo o texto deve ser digitado em tamanho 10,5. O
    espaçamento entre as linhas deverá ser simples. A expressão “resumo
    expandido” deve seguir a mesma tipografia das demais sessões primárias
    do trabalho.

    O texto do resumo expandido deve ser redigido em português e conter as
    seguintes seções (ver modelo): Introdução, Objetivos, Metodologia,
    Resultados e Discussão e Considerações Finais.  Deve apresentar no
    mínimo duas (02) e, no máximo, cinco (05) páginas contendo a mesma
    formatação em A5 do resumo e do abstract, bem como palavras-chave.

    \section*{Objetivos}
    Lorem ipsum dolor sit amet, consectetur adipiscing elit. Phasellus vitae
    dolor lacus. Ut accumsan vitae felis nec porttitor. Integer interdum
    fringilla feugiat. Nullam pulvinar sit amet tellus eget maximus. Donec
    sit amet magna eget justo semper fermentum vel eget velit.  In iaculis
    imperdiet mauris, ac ornare libero placerat non. Nulla libero lectus,
    ullamcorper ac ornare eget, pulvinar ac nulla. Curabitur vestibulum non
    nisl eget sagittis. Proin gravida lacus id eros bibendum interdum.
    Mauris ullamcorper elementum tortor sed consequat.  Integer tempus, est
    a lobortis vehicula, nisi mi fringilla augue, non semper leo metus in
    quam. Etiam in leo maximus, pulvinar mi eget, vehicula risus. Donec sed
    dui semper, dictum eros at, suscipit felis.

    \section*{Metodologia}
    Quisque efficitur dolor in lectus dapibus elementum. Nam ultrices
    blandit consectetur.  Nullam ultricies sit amet odio quis placerat.
    Aenean eget est elit. Maecenas et nulla dolor.  Orci varius natoque
    penatibus et magnis dis parturient montes, nascetur ridiculus mus. In
    pulvinar velit sed mi sagittis ornare. Aenean rutrum suscipit egestas.
    Phasellus pharetra eget ex in volutpat. Quisque eu arcu nunc. Vivamus
    arcu ligula, pharetra at rhoncus sit amet, pulvinar sed eros. Sed porta
    ipsum ipsum, et fermentum magna volutpat sed. Vivamus pharetra facilisis
    orci, sit amet luctus nisl pretium id. Sed consequat, arcu et congue
    pulvinar, risus enim aliquet purus, eget venenatis libero leo sit amet
    metus. Maecenas vitae elit sapien. Fusce mollis libero et gravida
    placerat. Proin ut quam quis justo aliquam dictum. Donec volutpat
    convallis suscipit. Vivamus metus nisl, placerat ac enim vitae, tempus
    ultricies odio.

    Aliquam ac vehicula arcu, non bibendum nulla. Morbi libero sem,
    imperdiet vel quam et, posuere tempus nunc. Maecenas dictum magna sit
    amet ligula facilisis commodo. Aliquam tellus diam, ornare vel elementum
    in, dignissim id purus. Ut at tortor non sem molestie euismod non at
    turpis. Phasellus vitae bibendum tellus. Suspendisse odio enim, faucibus
    eget congue quis, semper sit amet tortor. Sed ac lectus est.
    Pellentesque nec mattis mi, et varius dolor. Aliquam quis massa ac
    tellus malesuada sollicitudin. Maecenas ultrices risus massa, nec auctor
    risus sagittis id. Praesent a sapien nulla. Donec tincidunt, metus quis
    hendrerit facilisis, enim augue convallis elit, sed consequat lacus odio
    vitae magna.

    \section*{Resultados e Discussão}
    Nullam sed cursus leo. Donec commodo volutpat hendrerit. Fusce et tempus
    lectus, feugiat consequat est. Class aptent taciti sociosqu ad litora
    torquent per conubia nostra, per inceptos himenaeos. Nam quis cursus
    mauris, non tempus orci. Phasellus lobortis et mauris at vulputate. Sed
    nec nisl elementum lorem commodo gravida non a enim. Phasellus neque
    erat, aliquet ac ligula ac, maximus vestibulum sem. Vestibulum vel
    tincidunt turpis. Donec lacinia rutrum dolor dapibus bibendum. Mauris
    pharetra nibh nec tincidunt iaculis. Vivamus pharetra bibendum nisl eget
    blandit. In lobortis diam non justo eleifend, id lobortis ante
    fringilla.  Donec libero tortor, suscipit vestibulum vestibulum id,
    rutrum accumsan turpis. Phasellus sollicitudin luctus tincidunt.
    Suspendisse potenti. Nam semper metus et mi pharetra, in pretium ligula
    fermentum. Integer consectetur, orci non placerat feugiat, dui ex
    gravida augue, vel placerat ligula augue vel velit. Aliquam sollicitudin
    pellentesque congue. Donec vitae turpis in ante posuere posuere.
    Pellentesque eu justo leo. Donec quis elit vitae leo varius luctus quis
    eget justo.

    Nulla porta auctor vestibulum. Sed consectetur lacus molestie iaculis
    ullamcorper. Proin porta posuere massa a lacinia. Nunc a lacinia orci,
    non vehicula ante. Vestibulum ipsum velit, congue et neque aliquam,
    imperdiet ornare augue. Donec et congue sapien. Pellentesque consequat
    consectetur neque ut varius. In aliquam ex quis ante venenatis dapibus.
    Vivamus et imperdiet urna. Vestibulum quis nibh magna. In a congue
    lectus, eu sodales nunc. Suspendisse id.

    \section*{Considerações Finais}
    Lorem ipsum dolor sit amet, consectetur adipiscing elit. Phasellus vitae
    dolor lacus. Ut accumsan vitae felis nec porttitor. Integer interdum
    fringilla feugiat. Nullam pulvinar sit amet tellus eget maximus. Donec
    sit amet magna eget justo semper fermentum vel eget velit.  In iaculis
    imperdiet mauris, ac ornare libero placerat non. Nulla libero lectus,
    ullamcorper ac ornare eget, pulvinar ac nulla. Curabitur vestibulum non
    nisl eget sagittis. Proin gravida lacus id eros bibendum interdum.
    Mauris ullamcorper elementum tortor sed consequat.  Integer tempus, est
    a lobortis vehicula, nisi mi fringilla augue, non semper leo metus in
    quam. Etiam in leo maximus, pulvinar mi eget, vehicula risus. Donec sed
    dui semper, dictum eros at, suscipit felis.

    \UFSCimprimirPalavrasChave{Palavras-chaves}
    {\begin{inparaitem}[]\palavrasChavePortugues\end{inparaitem}}

  \end{otherlanguage*}

  \@ifundefined{imprimirenglishabstract}{}{\imprimirenglishabstract}

\else
  \@ifundefined{imprimirbrazilabstract}{}{\imprimirbrazilabstract}
  \@ifundefined{imprimirenglishabstract}{}{\imprimirenglishabstract}
\fi

\@ifundefined{imprimirfrenchabstract}{}{\imprimirfrenchabstract}
\@ifundefined{imprimirspanishabstract}{}{\imprimirspanishabstract}
\makeatother


  % Some tables of contents
  \ifforcedinclude\else
  {
    % q/disable-colorlinks-locally-or-just-for-the-toc
    \hypersetup{hidelinks}

    % inserir lista de figuras
    \ifforcedinclude\else
      \cleardoublepage{}
    \fi
    % q/list-of-figures-and-tables-when-there-are-no-figures-or-tables
    \whenlistisnotempty{\listfigurename}{%
      \UFSCaddToTextPreliminaryContent{\listfigurename}
      {
        % q/remove-spacing-between-per-chapter-figures-in-lof
        \renewcommand{\addvspace}[1]{}
        \listoffigures*
      }
    }{
      \pdfbookmark[0]{\listfigurename}{lof}
    }

    % inserir lista de quadros
    \ifforcedinclude\else
      \cleardoublepage{}
    \fi
    % q/list-of-figures-and-tables-when-there-are-no-figures-or-tables
    \whenlistisnotempty{\listofquadrosname}{%
      \UFSCaddToTextPreliminaryContent{\listofquadrosname}
      {
        % q/remove-spacing-between-per-chapter-figures-in-lof
        \renewcommand{\addvspace}[1]{}
        \listofquadros*
      }
    }{
      \pdfbookmark[0]{\listofquadrosname}{loq}
    }

    % inserir lista de tabelas
    \ifforcedinclude\else
      \cleardoublepage{}
    \fi
    % q/list-of-figures-and-tables-when-there-are-no-figures-or-tables
    \whenlistisnotempty{\listtablename}{%
      \UFSCaddToTextPreliminaryContent{\listtablename}
      {
        % q/remove-spacing-between-per-chapter-figures-in-lof
        \renewcommand{\addvspace}[1]{}
        \listoftables*
      }
    }{
      \pdfbookmark[0]{\listtablename}{lot}
    }

    % inserir códigos fonte (List of Listings `lol`)
    % q/latex-keeps-showing-minted-environment-as-figures-instead-of-listening
    \ifforcedinclude\else
      \cleardoublepage{}
    \fi
    % q/list-of-figures-and-tables-when-there-are-no-figures-or-tables
    \whenlistisnotempty{\lstlistlistingname}{%
      \UFSCaddToTextPreliminaryContent{\lstlistlistingname}
      {
        % q/remove-spacing-between-per-chapter-figures-in-lof
        \renewcommand{\addvspace}[1]{}
        \lstlistoflistings
      }
    }{
      \pdfbookmark[0]{\lstlistlistingname}{lol}
    }
  }
  \fi

  \ifforcedinclude\else
    \cleardoublepage{}
  \fi
  % imprime lista de siglas ordenado alfabeticamente
  % q/printnoidxglossaries-vs-printglossaries
  \pdfbookmark[0]{\ABNTEXlistadesiglasname}{acn}
  \printnoidxglossary[%
    style=UFSClong1,
    type=acronym,
    sort=letter,
    title={\ABNTEXlistadesiglasname\vspace{-0.2cm}}
  ]
  \cleardoublepage{}

  \ifforcedinclude\else
    \cleardoublepage{}
  \fi
  % imprime lista de símbolos na ordem em que aparece no texto
  % q/printnoidxglossaries-vs-printglossaries
  \pdfbookmark[0]{\ABNTEXlistadesimbolosname}{sbl}
  \printnoidxglossary[%
    style=UFSClong2,
    type=UFSCsimbolos,
    sort=use,
    title={\ABNTEXlistadesimbolosname}
  ]
  \cleardoublepage{}

  % How to remove the self-reference of the ToC from the ToC?
  % q/how-to-remove-the-self-reference-of-the-toc-from-the-toc
  \ifforcedinclude\else
    \cleardoublepage{}
  \fi

  \begin{KeepFromToc}
    % q/what-does-overfull-hbox-mean
    % q/how-to-avoid-using-sloppy-document-wide-to-fix-overfull-hbox-problems
    % q/adding-color-to-table-of-contents-and-section-headings
    {
      % q/disable-colorlinks-locally-or-just-for-the-toc
      \hypersetup{hidelinks}

      % q/underfull-vbox-badness-10000-with-memoir
      \raggedbottom

      % q/overfull-hbox-warning-for-toc-entries-when-using-memoir-documentclass
      % \makeatletter
        % \renewcommand{\@pnumwidth}{2em}
        % \renewcommand{\@tocrmarg}{3em}
      % \makeatother

      % q/memoir-mysterious-overfull-hbox-in-toc-when-mathptmx-is-used
      % \setlength{\cftchapternumwidth}{2.25em}

      % Add the table of contents to the brief table of contents
      % q/list-of-figures-and-tables-when-there-are-no-figures-or-tables
      \whenlistisnotempty{\contentsname}{%
        \UFSCaddToTextPreliminaryContent{\contentsname}
        \tableofcontents
      }{
        \pdfbookmark[0]{\contentsname}{toc}
      }
    }

  \end{KeepFromToc}

  % ELEMENTOS TEXTUAIS
  \clearpage
  \ABNTEXtextual
  \setlength\beforechapskip{50pt}
  \setlength\midchapskip{20pt}
  \setlength\afterchapskip{20pt}

  % PARTE
  \ifforcedinclude\else
    \part{\EngAndPor{Research}{Pesquisa}}
  \fi
  \label{primeira_parte}

  % Introdução (exemplo de capítulo sem numeração, mas presente no Sumário)
  % Copyright 2017 Evandro Coan
% Copyright 2012-2016 by abnTeX2 group at http://www.abntex.net.br/
%
% This work may be distributed and/or modified under the
% conditions of the LaTeX Project Public License, either version 1.3
% of this license or (at your option) any later version.
% The latest version of this license is in
%   http://www.latex-project.org/lppl.txt
% and version 1.3 or later is part of all distributions of LaTeX
% version 2005/12/01 or later.
%
% This work has the LPPL maintenance status `maintained'.
% The Current Maintainer of this work is the Evandro Coan.
%
% The last Maintainer of this work was the abnTeX2 team, led
% by Lauro César Araujo. Further information are available on
% https://www.abntex.net.br/
%
% This work consists of a bunch of files. But originally there were 3 files
% which are renamed as follows:
% Renamed the `abntex2-modelo-include-comandos` to `chapter_1.tex`
% Renamed the `abntex2-modelo-trabalho-academico.tex` to `intro.tex`
% Renamed the `abntex2-modelo-references.bib` to `modelo-ufsc-references.bib`
%
% This file was originally the main template file, however this main file was
% split into several new files, which are respectively drastically changed,
% except this files which contains most of the main documentation message.
%

% ------------------------------------------------------------------------
% abnTeX2: Modelo de Trabalho Academico (tese de doutorado, dissertacao de
% mestrado e trabalhos monograficos em geral) em conformidade com
% ABNT NBR 14724:2011: Informacao e documentacao - Trabalhos academicos -
% Apresentacao
% ------------------------------------------------------------------------

% The \phantomsection command is needed
% to create a link to a place in the document
% that is not a % figure, equation, table, section, subsection, chapter, etc.
% q/when-do-i-need-to-invoke-phantomsection
\phantomsection

% q/is-it-possible-to-keep-my-translation-together-with-original-text
\chapter{\EngAndPor{Introduction}{Introdução}}
\phantomsection

A Tabela~\ref{tab:a_table_formatacao_de_texto} mostra  informações do modelo de
teses da Biblioteca Universitária da UFSC (BU-UFSC).

% What does [t] and [ht] mean?
% q/what-does-t-and-ht-mean
%
% How can I get rid of the LaTeX warning: Float too large for page?
% q/how-can-i-get-rid-of-the-latex-warning-float-too-large-for-page
%
% "warning: Text page X contains only floats" How to suppress this warning?
% q/warning-text-page-x-contains-only-floats-how-to-suppress-this-warning
%
% Make a table span multiple pages
% q/make-a-table-span-multiple-pages
%
% How to make the longtable to work with centering & caption on memoir class?
% q/how-to-make-the-longtable-to-work-with-centering-caption-on-memoir-class
%
% How to fix this Package array Error: Only one column-spec allowed?
% q/how-to-fix-this-package-array-error-only-one-column-spec-allowed
%
% How to auto adjust my last table column width,
% and why is there Underfull \vbox badness on this table?
% q/how-to-auto-adjust-my-last-table-column-width-and-why-is-there-underfull-vbox/387251
\setlength\extrarowheight{2pt}
\begin{tabularx}{\linewidth}{>{\RaggedRight}p{3cm}|>{\arraybackslash}X}

\caption[Formatação do texto]{Formatação do texto}%
\label{tab:a_table_formatacao_de_texto} \\
\hline
\endfirsthead

% How to set font size of footnotes correctly in memoir?
% q/how-to-set-font-size-of-footnotes-correctly-in-memoir
\multicolumn{2}{p{\dimexpr\textwidth-2\tabcolsep\relax}}
{\UFSCcaptionSize\tablename~\thetable: 
  Formatação do texto (continuação) } \\
\hline
\endhead

% Set multicolumn width to default table width
% q/set-multicolumn-width-to-default-table-width
\hline
\multicolumn{2}{p{\dimexpr\textwidth-2\tabcolsep\relax}}
{\footnotesize continua na próxima página\protect\englishword{}}
\endfoot

\hline
\multicolumn{2}{p{\dimexpr\textwidth-2\tabcolsep\relax}}
{\ABNTEXfonte{O autor } }
\endlastfoot
    Cor                          & Branco - \englishword{}                                 \\ \hline
    Formato do papel             & A4                                                               \\ \hline
    Gramatura                    & 75                                                               \\ \hline
    Impressão                    & Frente e verso                                                   \\ \hline
    Margens                      & Direita e superior 3, Inferior e esquerda: 2.                    \\ \hline
    Cabeçalho                    & 0,7                                                              \\ \hline
    Rodapé                       & 0,7                                                              \\ \hline
    Paginação                    & Externa                                                          \\ \hline
    Alinhamento vertical         & Superior                                                         \\ \hline
    Alinhamento do texto         & Justificado                                                      \\ \hline
    Fonte sugerida               & Times New Roman                                                  \\ \hline
    Tamanho da fonte             & 12 para o texto incluindo os títulos das seções e subseções.
                                   As citações com mais de três linhas as legendas das ilustrações
                                   e tabelas, fonte 10.                                             \\ \hline
    Espaçamento entre linhas     & Um e meio (1,5)                                                  \\ \hline
    Espaçamento entre parágrafos & Anterior 0,0; Posterior 0,0                                      \\ \hline
    Numeração da seção           & As seções  primárias devem  começar  sempre em páginas ímpares.
                                   Deixar um espaço (simples) entre o título da seção e o texto e
                                   entre o texto e o título da subseção.                            \\ \hline

\end{tabularx}

\begin{figure}
    \caption{Exemplo de figura}
    \label{fig:ex01}
    \centering
    \includegraphics[width=\linewidth]{fig/ex01}
\ABNTEXfonte{o autor }
\end{figure}

Por exemplo, na \figref{fig:ex01}, tem-se...

\begin{figure}
    \caption{Exemplo de aquisição}
    \label{fig:tek0009}
    \centering
    \includegraphics[width=0.9\linewidth]{fig/tek0009}
    \ABNTEXfonte{o autor }
\end{figure}

Este documento e seu código-fonte são exemplos de referência de uso da classe
\textsf{abntex2} e do pacote \textsf{abntex2cite}. O documento
exemplifica a elaboração de trabalho acadêmico (tese, dissertação e outros do
gênero) produzido conforme a ABNT NBR 14724:2011 \emph{Informação e documentação
- Trabalhos acadêmicos - Apresentação}.

A expressão ``Modelo Canônico'' é utilizada para indicar que \abnTeX{} não é
modelo específico de nenhuma universidade ou instituição, mas que implementa tão
somente os requisitos das normas da ABNT. Uma lista completa das normas
observadas pelo \abnTeX{} é apresentada em \textcite{abntex2classe}.

Sinta-se convidado a participar do projeto \abnTeX{}! Acesse o site do projeto
em \url{http://abntex2.googlecode.com/}. Também fique livre para conhecer,
estudar, alterar e redistribuir o trabalho do \abnTeX{}, desde que os arquivos
modificados tenham seus nomes alterados e que os créditos sejam dados aos
autores originais, nos termos da ``The \LaTeX{} Project Public
License''\footnote{\url{http://www.latex-project.org/lppl.txt}}.

Encorajamos que sejam realizadas customizações específicas deste exemplo para
universidades e outras instituições --- como capas, folha de aprovação, etc.
Porém, recomendamos que ao invés de se alterar diretamente os arquivos do
\abnTeX{}, distribua-se arquivos com as respectivas customizações.
Isso permite que futuras versões do \abnTeX{}~não se tornem automaticamente
incompatíveis com as customizações promovidas. Consulte
\textcite{abntex2-wiki-como-customizar} par mais informações.

Este documento deve ser utilizado como complemento dos manuais do \abnTeX{}
\cite{abntex2classe,abntex2cite,abntex2cite-alf} e da classe \textsf{memoir}
\cite{memoir}.

Esperamos, sinceramente, que o \abnTeX{} aprimore a qualidade do trabalho que
você produzirá, de modo que o principal esforço seja concentrado no principal:
na contribuição científica.

Equipe \abnTeX{}

Lauro César Araujo


  % Capitulo com exemplos de comandos inseridos de arquivo externo
  \include{15b_chapter_1.tex}

  % Capitulo de revisão de literatura
  % The \phantomsection command is needed
% to create a link to a place in the document that is not a
% figure, equation, table, section, subsection, chapter, etc.
% q/when-do-i-need-to-invoke-phantomsection
\phantomsection{}

\chapter{\EngAndPor{Chapter Title}{Título do Capítulo}}

\section{\EngAndPor{Section Title}{Título da Seção}}

\subsection{\EngAndPor{SubSection Title}{Título da subSeção}}
\subsubsection{\EngAndPor{SubSubSection Title}{Título da subsubSeção}}

\EngAndPor{
  Here it can be written much stuff.
}{
  Aqui pode ser escrito várias coisas.
}


  % PARTE
  \ifforcedinclude\else
    \part{\EngAndPor{Implementation}{Implementação}}
  \fi%
  \label{segunda_parte}

  % Finaliza a parte no bookmark do PDF para que se inicie o bookmark na raiz
  % e adiciona espaço de parte no Sumário
  \ABNTEXphantompart{}

  % Conclusão (outro exemplo de capítulo sem numeração e presente no sumário)
  % The \phantomsection command is needed to create a link
% to a place in the document that is not a
% figure, equation, table, section, subsection, chapter, etc.
% q/when-do-i-need-to-invoke-phantomsection
\phantomsection{}

% ---
\chapter{\EngAndPor{Final Remarks}{Considerações Finais}}
\phantomsection{}

As conclusões devem responder às questões da pesquisa, em relação aos 
objetivos e às hipóteses, podendo apresentar recomendações e sugestões para 
trabalhos futuros.


  % ELEMENTOS PÓS-TEXTUAIS
  \ABNTEXpostextual{}
  \setlength\beforechapskip{0pt}
  \setlength\midchapskip{15pt}
  \setlength\afterchapskip{15pt}

  % Referências bibliográficas
  \begingroup
    % q/how-to-modify-line-spacing-per-entry-of-bibliography
    % q/how-can-i-put-more-space-between-bibliography-entries-biblatex
    \setlength\bibitemsep{\baselineskip}
    \ifadvisor\else
      \linespread{1.18}\selectfont
    \fi

    % q/using-bibtex-to-make-a-list-of-references-without-having-citations-in-the-body
    % \nocite{*}
    \printbibliography[title=\EngAndPor{REFERENCES}{REFERÊNCIAS}]
  \endgroup

  % Glossário, consulte o manual da classe abntex2
  % \ifforcedinclude\else
  %   \glossary
  % \fi

  \begin{abntexEnvApendicesenv}
    % Imprime uma página indicando o início dos apêndices
    \ifforcedinclude\else
      \ABNTEXpartapendices{}
    \fi
    \setlength\beforechapskip{50pt}
    \setlength\midchapskip{20pt}
    \setlength\afterchapskip{20pt}

    % How to fix the Underfull \vbox badness has occurred
% while \output is active on my memoir chapter style?
% q/how-to-fix-the-underfull-vbox-badness-has-occurred-while-output-is-active-on-m

\EngAndPor
{\chapter[Page not filled]
  {Since this page is not being completely filled, it is generating the bottom
   bottom of the page}
}
{\chapter[Página não gerada]
  {Como esta página não está sendo completamente preenchida, ele está gerando a
   caixa inferior inferior da página}
}

% Multiple-language document
% - babel - selectlanguage vs begin/end{otherlanguage}
% q/multiple-language-document-babel-selectlanguage-vs-begin-endotherlanguage
\begin{otherlanguage*}{english}

\UFSCshowFont

1. How to display the font size in use in the final output,
2. How to display the font size in use in the final output,
3. How to display the font size in use in the final output,
4. How to display the font size in use in the final output,
5. How to display the font size in use in the final output,
6. How to display the font size in use in the final output,
7. How to display the font size in use in the final output,
8. How to display the font size in use in the final output,
9. How to display the font size in use in the final output,

% As this page is not being completely filled,
% it is generating the page bottom bad box.
% Fix Underfull \vbox (badness 10000) has occurred while \output is active
%
% \flushbottom vs \raggedbottom
% q/flushbottom-vs-raggedbottom
\newpage

\section[Some encoding tests]{\UFSCshowFont}

1. How to display the font size in use in the final output,
2. How to display the font size in use in the final output,
3. How to display the font size in use in the final output,
4. How to display the font size in use in the final output,
5. How to display the font size in use in the final output,
6. How to display the font size in use in the final output,

7. How to display the font size in use in the final output,
8. How to display the font size in use in the final output,
9. How to display the font size in use in the final output,
10. How to display the font size in use in the final output,
11. How to display the font size in use in the final output,
12. How to display the font size in use in the final output,

\subsection{\UFSCshowFont}

1. How to display the font size in use in the final output,
2. How to display the font size in use in the final output,
3. How to display the font size in use in the final output,
4. How to display the font size in use in the final output,
5. How to display the font size in use in the final output,
6. How to display the font size in use in the final output,

7. How to display the font size in use in the final output,
8. How to display the font size in use in the final output,
9. How to display the font size in use in the final output,
10. How to display the font size in use in the final output,
11. How to display the font size in use in the final output,
12. How to display the font size in use in the final output,

\subsubsection{\UFSCshowFont}

1. How to display the font size in use in the final output,
2. How to display the font size in use in the final output,
3. How to display the font size in use in the final output,
4. How to display the font size in use in the final output,
5. How to display the font size in use in the final output,
6. How to display the font size in use in the final output,

7. How to display the font size in use in the final output,
8. How to display the font size in use in the final output,
9. How to display the font size in use in the final output,
10. How to display the font size in use in the final output,
11. How to display the font size in use in the final output,
12. How to display the font size in use in the final output,

\ABNTEXsubsubsubsection{\UFSCshowFont}

1. How to display the font size in use in the final output,
2. How to display the font size in use in the final output,
3. How to display the font size in use in the final output,
4. How to display the font size in use in the final output,
5. How to display the font size in use in the final output,
6. How to display the font size in use in the final output,
7. How to display the font size in use in the final output,

8. How to display the font size in use in the final output,
9. How to display the font size in use in the final output,
10. How to display the font size in use in the final output,
11. How to display the font size in use in the final output,
12. How to display the font size in use in the final output,

Lipsum me [31-35]

\end{otherlanguage*}

  \end{abntexEnvApendicesenv}

  \begin{abntexEnvAnexos}
    % Imprime uma página indicando o início dos anexos
    \ifforcedinclude\else
      \ABNTEXpartanexos{}
    \fi
    \setlength\beforechapskip{50pt}
    \setlength\midchapskip{20pt}
    \setlength\afterchapskip{20pt}

    % How to fix the Underfull \vbox badness has occurred
% while \output is active on my memoir chapter style?
% q/how-to-fix-the-underfull-vbox-badness-has-occurred-while-output-is-active-on-m

\chapter{\EngAndPor{Article published in SOBRAEP magazine}{Artigo publicado}}

% Multiple-language document
% - babel - selectlanguage vs begin/end{otherlanguage}
% q/multiple-language-document-babel-selectlanguage-vs-begin-endotherlanguage
\begin{otherlanguage*}{english}

% An environment for setting \emergencystretch locally
% q/an-environment-for-setting-emergencystretch-locally
{
    \setlength{\emergencystretch}{10pt}
    \section[English guidelines for publication]
    {English guidelines for publication - TITLE HERE (14 PT TYPE SIZE,
     UPPERCASE, BOLD, CENTERED)}
}
    \noindent\textbf{Abstract:}
    The objective of this document is to instruct the authors about the
    preparation of the manuscript for its submission to the Revista Eletrônica
    de Potência (Brazilian Power Electronics Journal).~The authors should use
    these guidelines for preparing both the initial and final versions of their
    paper. Additional information about procedures and guidelines for
    publication can be obtained directly with the editor, or through the web
    site \url{http://www.sobraep.org.br/revista}. This text was written
    according to these guidelines

\end{otherlanguage*}

% What is a “Overfull \hbox (9.89561pt too wide)”?
% q/what-is-a-overfull-hbox-9-89561pt-too-wide
interwordspace: \the\fontdimen2\font{}

interwordstretch: \the\fontdimen3\font{}

emergencystretch: \the\emergencystretch\par\relax

\modifiedincludepdf{-}{ArtigoSOBRAEP}{fig/SOBRAEP.pdf}{0.9}

    % How to fix the Underfull \vbox badness has occurred
% while \output is active on my memoir chapter style?
% q/how-to-fix-the-underfull-vbox-badness-has-occurred-while-output-is-active-on-m

\lang
{\chapter[Sample example]
 {How to display the font size in use in the final output}
}
{\chapter[Anexo exemplo]
{Como exibir o tamanho da fonte em uso na saída final}
}

% Multiple-language document
% - babel - selectlanguage vs begin/end{otherlanguage}
% q/multiple-language-document-babel-selectlanguage-vs-begin-endotherlanguage
\begin{otherlanguage*}{english}

\UFSCshowFont

1. How to display the font size in use in the final output,
2. How to display the font size in use in the final output,
3. How to display the font size in use in the final output,

\section[Some encoding tests]{\UFSCshowFont}

1. How to display the font size in use in the final output,
2. How to display the font size in use in the final output,
3. How to display the font size in use in the final output,
4. How to display the font size in use in the final output,
5. How to display the font size in use in the final output,
6. How to display the font size in use in the final output,

7. How to display the font size in use in the final output,
8. How to display the font size in use in the final output,
9. How to display the font size in use in the final output,
10. How to display the font size in use in the final output,
11. How to display the font size in use in the final output,
12. How to display the font size in use in the final output,

\subsection{\UFSCshowFont}

1. How to display the font size in use in the final output,
2. How to display the font size in use in the final output,
3. How to display the font size in use in the final output,
4. How to display the font size in use in the final output,
5. How to display the font size in use in the final output,
6. How to display the font size in use in the final output,

7. How to display the font size in use in the final output,
8. How to display the font size in use in the final output,
9. How to display the font size in use in the final output,
10. How to display the font size in use in the final output,
11. How to display the font size in use in the final output,
12. How to display the font size in use in the final output,

\subsubsection{\UFSCshowFont}

1. How to display the font size in use in the final output,
2. How to display the font size in use in the final output,
3. How to display the font size in use in the final output,
4. How to display the font size in use in the final output,
5. How to display the font size in use in the final output,
6. How to display the font size in use in the final output,

7. How to display the font size in use in the final output,
8. How to display the font size in use in the final output,
9. How to display the font size in use in the final output,
10. How to display the font size in use in the final output,
11. How to display the font size in use in the final output,
12. How to display the font size in use in the final output,

\subsubsubsection{\UFSCshowFont}

1. How to display the font size in use in the final output,
2. How to display the font size in use in the final output,
3. How to display the font size in use in the final output,
4. How to display the font size in use in the final output,
5. How to display the font size in use in the final output,
6. How to display the font size in use in the final output,
7. How to display the font size in use in the final output,

8. How to display the font size in use in the final output,
9. How to display the font size in use in the final output,
10. How to display the font size in use in the final output,
11. How to display the font size in use in the final output,
12. How to display the font size in use in the final output,

Lipsum me [55-65]

\end{otherlanguage*}

  \end{abntexEnvAnexos}

  % INDICE REMISSIVO
  \ifforcedinclude\else
    \ABNTEXphantompart{}
    \printindex
  \fi

\end{document}
